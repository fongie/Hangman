\documentclass[a4paper]{scrartcl}
\usepackage[utf8]{inputenc}
\usepackage[english]{babel}
\usepackage{graphicx}
\usepackage{lastpage}
\usepackage{pgf}
\usepackage{wrapfig}
\usepackage{fancyvrb}
\usepackage{fancyhdr}
\pagestyle{fancy}

\usepackage[colorlinks=true,linkcolor=violet]{hyperref}
\usepackage[figure]{hypcap} %jump to img instead of caption text

% Create header and footer
\headheight 27pt
\pagestyle{fancyplain}
\lhead{\footnotesize{Network Programming, ID1212}}
\chead{\footnotesize{Homework 1: Hangman}}
\rhead{}
\lfoot{}
\cfoot{\thepage\ (\pageref{LastPage})}
\rfoot{}

% Create title page
\title{Homework 1: Hangman}
\subtitle{Network Programming, ID1212}
\author{Max Körlinge, korlinge@kth.se}
\date{11/8-2018}

\begin{document}

\maketitle

\section*{Tips for Report Writing}
\textbf{REMOVE THIS SECTION BEFORE SUBMITTING THE REPORT.}\\

\noindent \textit{The target audience has exactly the same skills as the author, except they do not know anything at all about the specific program described in the report.} \\

Consider the following:

\begin{itemize}
  \item \textbf{The report must be \textit{centered around the requirements}. Which are they (Introduction), how did you work to meet them (Method), what is the solution that meets them (Result), and how can you be sure they are met (Discussion). This is the IMRaD method.}

  \item \textbf{The report must show that you have done the work yourself and that you have understood what you have done. Both of these goals are met by carefully explaining the source code.}

  \item Is spelling and grammar correct? Is spoken language avoided?

  \item Does the report have a good structure with sections, subsections and paragraphs?

  \item Is the solution clearly explained? Will the reader understand the program? What would you yourself want to know if you read about the program, is that included in the report?

  \item Is the solution analyzed and evaluated? Are important properties of the program explained? Should there have been more extensive evaluation?

  \item Is the text clarified with images and/or other figures, and with links to the code in your Git repository? Remember that all figures (images, tables, graphs, code listings, etc) shall be numbered and have a short explaining text.
\end{itemize}

\section{Introduction}

\textbf{This section tells \textit{what} are you going to do.} \\

\noindent Explain the task and the requirements on the solution. If you wrote the program together with other students, list their names and email addresses here.

\noindent The assignment was to develop a distributed application, in this case, a game of Hangman. The requirements were:

\begin{itemize}
    \item Communicate using TCP blocking sockets between client and server.
    \item No state must be stored on the client.
    \item Server can only send state, not complete parts of the user interface.
    \item The user interface must be informative, so the player knows what is going on.
    \item Client and server must be multithreaded, for a better user experience and the ability to handle several clients at once.
    \item The programmed must be designed using a layered architecture and object-oriented design principles.
\end{itemize}

The program was fully written by the author of this report.

\section{Literature Study}

This section must prove that you collected sufficient knowledge before starting development, instead of just hacking away without knowing how to complete the task. State what you have read and briefly summarize what you have learned.

To prepare for this assignment, all video material provided by the course on Sockets and Stream handling was viewed together with the code of the sample programs.

\section{Method}

\textbf{This section tells \textit{how} you solved the task.} \\

\noindent Explain how you worked when solving the task and how you evaluated that your solution met the requirements. Mention development tools and IDEs you used. \textit{Do not explain your solution and do not refer to code}, that belongs to the \textit{Result} section.

\noindent The program was written in Java using the IntelliJ IDE. To simplify running the programs outside of the IDE, the client and server were split into two programs and Maven was used to be able to generate an executable JAR for both the server and the client program. Before writing any part of the program, a preliminary layered design was created as a directory tree in each program.

To facilitate developing the net layer by knowing exactly what would be sent, the game logic for Hangman was developed first on the client and server by hard coding dummy code in the net layer. This process took into account the requirements regarding state on the client and server side. After game logic was finished, the net layer concerning TCP communication was developed. After having a solution with TCP communication, the threading parts were added, to meet all requirements.

\section{Result}

\textbf{This section explains \textit{the result} of what you did.} \\

\noindent Present your solution and prove that it meets the requirements listed in the homework specification. Make sure to  \textit{clearly explain how you met each requirement}. You can, for example, write a separate subsection for each requirement, where you mention the requirement and explain how you met it. You can prove that a requirement is met by, for example, explaining source code related to the requirement. If so, include a link to that particular code in the Git repository, or clearly tell which method and class you intend. You can also prove that a requirement is met by including, and explaining, screenshots of the user interface.

Also, prove that you participated in writing the program, and that you understand it in detail. You prove this by explaining essential parts of the source code.

\noindent The complete source code can be found at \href{https://github.com/fongie/Hangman}{https://github.com/fongie/Hangman}.

\begin{itemize}
    \item To communicate using TCP sockets, one can decide on a text-based protocol or an object-based protocol. In this case, an object-based protocol was used. The Java net package was used to create a listening server socket in the server program, and an internet socket on the client side. These communicate using streams. The general streams provided by the sockets are then wrapped in streams handling Objects, where data transfer objects (DTO) that represent are sent and received. The client sends Guess objects which the server receives, and the server sends StatusReport objects which the client receives.
    \item Since no state was allowed to be stored on the client, the client simply receives \href{StatusReport}{https://github.com/fongie/Hangman/blob/master/hangmanserver/src/main/java/DTO/StatusReport.java} objects from the server, as explained above, which include game information such as the number of letters in the word, the current score, remaining attempts, and letters which have been correctly guessed so far. Note that no information is sent that can be able to cheat, such as the actual word that is to be guessed.
    \item The server was not allowed to send parts of the user interface, which is why the StatusReport object, explained above, only concerns primitives and nothing that can be directly used in the user interface on the client.
    \item For the user interface to be informative, the client program takes StatusReport objects that it receives and prints them prettily to the screen so that the user knows the current state of the game and sees a visual representation of the hangman game, which is the correct number of underscores and the previously guessed correct letters filled in at the right place. To achieve the presentation of the word and the guessed letters, a new DTO was introduced to represent correctly guessed letters and their position, which are sent by the server in the StatusReport and can be looped over by the client to fill in the correct letters guessed so far, \href{LetterPosition}{https://github.com/fongie/Hangman/blob/master/hangmanserver/src/main/java/DTO/LetterPosition.java}.
    \item To be able to handle multiple clients, the server starts a new thread for each connected client. The new thread runs an object from the class \href{Client}{https://github.com/fongie/Hangman/blob/master/hangmanserver/src/main/java/net/Client.java}, which starts a new game and handles all communication with that particular client. For a seamless user experience in the client, all socket communication is done in threads separate from the main user interface (console) thread. This is done using the common thread pool and the java concurrent package in the client program's \href{Controller}{https://github.com/fongie/Hangman/blob/master/hangmanclient/src/main/java/contr/Controller.java}. This means that while communication is being made between client and server, the user interface can still be interacted with on the client side.
    \item To comply with object-oriented design principles and a layered architecture, the programs were structured before-hand. The client consists of a classic View-Controller-Net architecture, where the View handles the user interface, the Controller handles the threading, and the Net package handles communication with the server. All function calls are made from the top down. One tricky part of this is how to handle the response from the server. This is done by having a inner class in the View be sent down to the Controller layer as a Consumer, which means that when the Controller initiates the server communication, the response will be consumed (accepted) by the Consumer, which can then format and print the response to the user interface, without having to receive an actual function call from the lower layer. The client program also contains a DTO package, for data objects, and a Startup package, to start the program.

        The server consists of merely two working layers, the Net and Model layers. There is also a DTO layer for the data objects, and a Startup layer. In the preliminary design plan, there was also a Controller layer, but since this layer only ended up forwarding function calls and not containing anything at all, it was removed, and the net layer was allowed to directly construct a new Game in the Model layer. This still means that the packages and classes are well encapsulated, cohesive, and that all function calls are made from the top down, from the Net to the Model layer.

\end{itemize}

\pagebreak

\section{Discussion}

\textbf{This section \textit{analysis} the result presented in the previous section.} \\

\noindent Summarize the requirements and \textit{clearly state which of them you have met}. This might seem like a meaningless repetition of the Result section, but it is not. Here, you shall not prove that the requirements are met, just briefly state what you achieved, that is, which requirements where met. This summary is very useful when assessing your report. Also, explain what lessons you learned, what problems you faced, and how the problems were solved. Should something have been done differently?

This assignment was to develop a distributed application using blocking sockets, and use threading to achieve a seamless user experience while being able to entertain several clients. The application was a game of Hangman which was required to have a layered architecture and an informative user interface. As presented above, all the basic requirements were met. There was an optional task to implement a length header to each socket message, but this was not implemented.

Since I have not worked with either socket communication or threading before, this was a learning experience on those subjects. Somewhat surprisingly, there were not many problems met with, and the development went smoothly - not very usual. This smoothness was probably due to the comprehensive lecture material, being more proficient at how to structure and develope programs after three years at university, and a bit of luck.

Given more time, I could improve the program to contain more user interaction, for example to start the game, receive a message on winning or losing instead of just the score being adjusted, to display what the word was when you fail, and to develop a graphical user interface. Since there were no requirements on these quality-of-life aspects, I chose to spend my time on studying other things instead.

\section{Comments About the Course}

I spent about 8 hours on the lecture material, about 12 hours on the assignment (including failed header-implementing-time), and about 2 hours to write the report, making 22 hours in total.

\end{document}
